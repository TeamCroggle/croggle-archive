\chapter{Integrationstests}
\section{Controller}
Für die Überprüfung der fehlerlosen Zusammenarbeit der voneinander abhängigen Systemkomponenten wurde eine Reihe weiterer Testfälle geschrieben.
Dazu gehören vor allem Tests, die mittels des TestHelpers eine Instanz der AlligatorApp beziehen, da dieser die komplette Applikation in einem "`headless"' Modus (d.h. ohne grafische Ausgabe) aufbaut und bereitstellt.
Viele der Controller konnten so bereits in einer nahezu "`realen"' Umgebung getestet werden, u.a.
\begin{description}
	\item[ProfileController] der ProfileControllerTest beinhaltet Sequenzen zur Verifizierung der Codebereiche, die Informationen zu Profilen in der Datenbank speichern.
		Hauptsächlich werden hier Profile angelegt, verändert und gelöscht.
		Aber auch das Setzen und Auslesen des aktuell aktiven Profils mittels libGdx' ApplicationPreferences wird hier überprüft.
		Hinzu kommen noch Tests zur Benachrichtigungsübermittlung bei Änderungen am derzeitigen Profil an andere Programmteile und die Validierung von Nutzerdaten.
	\item[SettingController] wird in SettingControllerTest im Zusammenspiel mit dem ProfileManager und in diesem Zusammenhang auch der Datenbank getestet.
	\item[StatisticController] auch dieser Controller wird in StatisticControllerTest auf die fehlerfreie Zusammenarbeit mit dem ProfileController getestet
	\item[AchievementController] der entsprechende Testfall interagiert direkt mit dem entsprechenden Datenbankmanager bzw. über den PersistenceManager
\end{description}


\section{Simulator}
Desweiteren befassen sich mehrere Testfälle in SimulatorTest.java mit der fehlerfreien Funktionsweise des Simulators.
Dabei wird unter anderem folgendes betrachtet:
\begin{description}
	\item[Validierung \& Lastgrenzen] Es existiert ein eigenes Unterpaket (operations.validation) um vom Nutzer eingegebene Konstellationen vor dem Simulationsbeginn auf Validität zu prüfen.
		Dass die darin enthaltenen Validierer unter den Bedingungen während der Simulation funktionieren, wird im SimulatorTest überprüft.
		Die Einhaltung der Lastgrenzen bei einer langwierigen Simulation wird durch testStackOverflowIssue getestet.
		Dabei dürfen keine Fehler passieren, abgesehen von jenen, die bei der Überschreitung der 300-Alligatoren Grenze auftreten.
	\item[Evaluation] Die Validierung von korrekten Ergebnissen der Simulation wird in mehreren Tests durchgeführt und beinhaltet somit auch die Überprüfung einer ganzen Reihe von "`board.operations"', die an der Evaluation beteiligt sind.
		Dazu zählen u.a.
		\begin{itemize}
			\item CopyConstellation
			\item ExchangeColor
			\item FindEating
			\item uvm...
		\end{itemize}
\end{description}


\section{Renderer}
Das wohl größte Modul in Bezug auf Komplexität und Vernetzung mit anderen Subsystemen stellt der BoardActor im "`renderer"'-Paket dar.
Der BoardActorTest prüft in dieser Hinsicht das Verhalten des Gesamtsystems bei einer sehr langen Simulation.
Dafür nimmt es den expandierenden Lambda-Term "`$\lambda$g.($\lambda$x.g (x x)) ($\lambda$x.g (x x))"' für eine Simulation und meldet den BoardActor als BoardEventListener an deren Simulator an.
Bei der anschließenden Simulation muss der BoardActor zuerst die Ausgangskonstellation darstellen ("`layouten"') und anschließend auf alle Änderungen an der Kosntellation reagieren.
Dies geschieht unter Zuhilfenahme der Funktionen des "`layout"'-Pakets.
