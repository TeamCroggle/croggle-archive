\chapter{Feldtest}
Zusätzlich zu den in den vorherigen Kapiteln beschriebenen Tests wurde ein Feldtest durchgeführt.
Für diesen wurde die Applikation von 25 Personen im Alter zwischen 13 und 56 Jahren getestet und anschließend jeweils ein Fragebogen ausgefüllt.
Unter den Testpersonen befanden sich sowohl Menschen mit, als auch ohne Vorkenntnissen bezüglich des Lambdakalküls und des Umgangs mit Tablets bzw. Smartphones.
Im Folgenden werden die Erkenntnisse, die wir durch den Feldtest erhalten haben, zusammengefasst und die daraus resultierenden
Änderungen an der Applikation aufgeführt.

\section{Auswertung der Tests}
\subsection{Erreichen und Bedienen von Statistiken, Achievements und Einstellungen}
Sowohl die Statistiken, als auch die Achievements und die Einstellungen, sind laut den Testpersonen sehr intuitiv zu bedienen.

\subsection{Verwalten von Benutzerprofilen}
Benutzerprofile sind laut den Tests einfach anzulegen und zu wechseln, obwohl es sehr vereinzelt vorkam, dass Testpersonen den Button zum Wechseln eines Profils nicht als solchen erkannten.
Das Modifizieren der Benutzerprofile wurde ebenfalls als sehr intuitiv bewertet.

\subsection{Wartezeiten}
Es traten keine Wartezeiten auf, die den Testpersonen zu lange erschienen.

\subsection{Spielprinzip und Level}
\begin{description}
\item[Einfärbe- und Einfüge-Level] Das Spielprinzip dieser Leveltypen wurde neutral, mit einer leichten Tendenz zum Positiven, bewertet.
Als größter Kritikpunkt wurden die Tutorials angeführt. Diese waren, vor allem zu Beginn, unverständlich.

\item[Multiple-Choice-Level] Multiple-Choice-Level wurden vom Spielprinzip her etwas schlechter bewertet als die anderen Leveltypen.
Kritikpunkte waren auch hier die Tutorials und zusätzlich der unübersichtliche Aufbau des Bildschirms.

\item[Hinweise] Die Hinweise wurden von fast allen Testpersonen benutzt und als hilfreich bewertet.

\item[Schwierigkeit der Level] Die Schwierigkeit der Level wurde von den Testpersonen als angemessen bewertet.
\item[Farben]Es wurde von vielen Testpersonen bemerkt, dass sich manche der verwendeten Farben sehr ähnlich sehen.
Dieses Problem trat im Farbenblindmodus nicht auf.
\end{description}


\section{Resultierende Änderungen}
\begin{description}
\item[Tutorial] Die Tutorials wurden noch einmal von Grund auf überarbeitet.
\item[Multiple-Choice-Screen] Es wurden kleinere Änderungen in der Darstellung der Antwortmöglichkeiten vorgenommen.
So wurden diese näher zusammengerückt, sodass der Bildschirm nun übersichticher erscheint.
\item[Farben] Bei den Farben wurde sowohl die Farbpalette, als auch die Auswahl der Farben in den Leveln angepasst.
\end{description}
