\chapter{Einleitung}
Nach dem Abschluss der Implementierungsphase bot Croggle bereits eine Vielfalt an Funktionen, welche bereits zu großen Teilen fehlerfrei zusammenspielten.
Trotzdem fühlte sich das Produkt an einigen Stellen noch unfertig an.
So kam es etwa zu gelegentlichen Abstürzen oder inkonsistentem Verhalten; auch die Bedienung und die Oberfläche waren noch etwas "`kantig"'.


In der nun abgeschlossenen Testphase wurde jetzt die Gelegenheit wahrgenommen, möglichst alle Störfaktoren zu beheben und an der ein oder anderen Stelle sogar noch größere Verbesserungen zu integrieren.
Alle Änderungen werden im Folgenden im Detail aufgeführt und erläutert.
Zudem wurde eine umfangreiche Testbasis geschaffen, um die Qualität der Software nicht nur zu belegen, sondern auch für die Zukunft zu sichern.
Bei der Erstellung der Tests wurde sich hierzu am sogenannten "`V-Modell"' orientiert, wodurch sich die Testfälle in Kategorien einteilen lassen, wie sie sich als einige der Kapitelüberschriften in diesem Dokument wiederfinden.
