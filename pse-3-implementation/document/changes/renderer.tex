\section{Package: ui.renderer}

Für das "`renderer"' Paket wurde während der Entwurfsphase nur ein oberflächliches Konzept erstellt.
Daher kam es zu umfangreichen Erweiterungen und teilweise auch zu Änderungen am Entwurf.

\subsection{Änderungen gegenüber dem Entwurf}

Die wichtigste Änderung am Entwurf betrifft die Vererbungshierarchie der BoardActors. 
Während der Entwurf noch einen ParentActor vorsah, empfahl es sich während der Implementierung eine gemeinsame Oberklasse für gefärbte BoardObjects zu haben.
Dies liegt an der ähnlichen vorgehensweise beim Zeichnen der unterschiedlichen Farben.
An die Stelle des ParentActors tritt somit der ColoredBoardObjectActor.

\subsection{Neuerungen gegenüber dem Entwurf}

Das Erstellen und die Synchronhaltung einer Repräsentation zu einem Board ist ein komplexes Thema, welches im Entwurf noch nicht detailliert beachtet wurde.
Für den hierfür erforderlichen Code wurde daher nachträglich die mit dem Präfix ActorLayout versehenen Klassen angelegt.
Diese werden hier kurz erläutert:

\begin{description}
\item[ActorLayout]
	Kapselt die für die Darstellung benötigten Actors. 
	Beinhaltet zudem Methoden für den Lookup BoardObject $\leftrightarrow$ BoardObjectActor, die Konfiguration, mit der es erstellt wurde und gecachete Statistiken zu dem Layout.
\item[ActorLayouter]
	Stark konfigurierbare, abstrakte Elternklasse, um die Daten eines Layouts zu generieren.
	Dies sind vor allem Positionen und Größen von BoardObjectActors.
	Durch das Anbieten von Schablonenmethoden und Callbacks kann der gleiche Code genutzt werden, um Layouts zu kreieren, als auch, um sie an Änderungen im Board anzupassen.
\item[ActorLayoutConfiguration]
	Kapselt alle Informationen, die der ActorLayouter zum Erstellen der Informationen benötigt. Dazu gehören z.B. Paddings, der vertikale Skalierungsfaktor etc.
\item[ActorLayoutStatistics]
	Ort, um Statistiken eines Layouts zu cachen, wie z.B. die Breitenkarte (width map), die jedem Actor die Breite des von ihm bewurzelten Unterbaumes zuweist.
\item[ActorLayoutBuilder]
	Helferklasse, um zu einem gegebenen Board ein Layout zu erstellen.
	Der Hauptteil des dafür benötigten Codes wird von ActorLayouter geerbt.
	Lediglich Schablonenmethoden zum Erzeugen von zu layoutenden BoardObjectActors werden implementiert.
\item[ActorLayoutFixer]
	Helferklasse, um eine Sammlung von ActorDeltas zu produzieren, die angewandt werden müssen, um ein gegebenes Layout an ein gegebenes Board anzupassen.
	Die Deltas können anschließend leicht animiert und umgesetzt werden (siehe hierzu BoardActorBoardChangeAnimator)
	Der Hauptteil des dafür benötigten Codes wird von ActorLayouter geerbt.
	In der Schablonenmethode zur Erzeugung von zu layoutenden BoardObjectActors werden immer wieder die gleichen Actors übergeben.
	Danach müssen in einer callback Methode lediglich Deltas zu den bereits im Layout vorhandenen Actors gebildet werden.
\item[ActorDelta]
	Fasst Änderungen an einem Actor zusammen.
	Diese werden vom BoardActorBoardChangeAnimator benutzt, um das Layout an den Zustand des Boards anzupassen.
\end{description}

Desweiteren ist anzumerken, dass der BoardActor zu einer hoch integrierten grafischen Komponente gewachsen ist.
Zu seinem Funktionsumfang gehören:
\begin{description}
\item[BoardActorZoomAndPan] 
	Zooming und Panning Funktionalitäten für das dargestellte Layout
\item[BoardActorLayoutEditing] 
	Boardbearbeitungsfunktionalitäten, wie z.B. Farbauswahl Popups für ungefärbte Actos oder Drag and Drop.
\item[BoardActorBoardChangeAnimator] 
	Animationen bei eingehenden BoardEvents
\end{description}

Alle diese Funktionen sind jedoch gesondert in Module gekapselt ohne sich gegenseitig direkt zu referenzieren.
Dazu bietet der BoardActor ein erweitertes Interface, welches aber nur im Paket "`renderer"' zur Verfügung steht.


Zum Schluss seien noch die verbliebenen, nicht im Entwurf enthaltenen Klassen genannt und kurz erklärt:
\begin{description}
\item[BoardObjectActorDragging]
	Helferklasse für alle Akteure, die bei Drag\&Drop mit BoardObjectActors partizipieren möchten.
	Für diesen Zweck werden für alle drei BoardObjectActor Typen Actors erzeugt, die in libgdx' DragAndDrop als DragActor, ValidDragActor und InvalidDragActor gesetzt werden können.
\item[ColorSelectorPopup]
	Dedizierte grafische Komponente, mit der BoardActorLayoutEditing dem Nutzer Eingabemöglichkeiten zur Änderung von Farben von ColoredBoardObjects bereitstellt.
\item[ObjectBar]
	Grafisches Element, welches eine Leiste darstellt, aus der neue BoardObjectActors mittels Drag\&Drop gezogen und auf dem Spielfeld platziert werden können.
	Außerdem werden bereits vorhandene BoardObjectActors, die auf die Leiste gezogen werden, gelöscht.
\item[PlaceHolderActor]
	Eine Erweiterung des BoardObjectActors, welcher keine grafische Repräsentation bietet, aber dafür als Platzhalter dient.
	Werden vom Nutzer andere BoardObjectActors auf den Platzhalter geschoben, so wird die zugrundeligende Konstellation so modifiziert, dass der verschobene Actor den Platz des Platzhalters einnimmt.
\item[TreeGrowth]
	Enum (Aufzählung) von Richtungen, in die im Allgemeinen `"gewachsen"' werden kann. 
	Wird hauptsächlich zur Angabe verwendet, in welche Richtung der Layout Baum wachsen soll.
\item[WorldPane]
	Paketsichtbare, grafische Komponente, mit der der BoardActor eine Übersetzung von Bildschirm- bzw. Szenenkoordinaten in die Koordinaten der gezoomten und verschobenen Layout-Welt realisiert.
\end{description}
