\chapter{Einführung}

Die Applikation "`Croggle"' soll Grundschülern schon früh grundlegende Prinzipien und Funktionsweisen der funktionalen Programmierung, speziell dem untypisiertem Lambda-Kalkül, vermitteln.
In diesem Entwurfsdokument werden sowohl die Designkonzepte, als auch die dahinterstehende Architektur der Applikation dargelegt.

Den wichtigsten Teil des Dokumentes bildet die Klassendokumentation, in der alle Klassen und ihre Methoden aufgelistet und beschrieben sind.
Dies wird ergänzt durch das Klassendiagramm, das sowohl Attribute und Methoden, als auch die Beziehungen der Klassen untereinander darstellt.
Weiterhin werden typische Programmabläufe, und die Interaktion der Klassen dabei, durch Sequenzdiagramme beschrieben.
Außerdem wird auf andere Punkte, die für den Aufbau der Applikation wichtig sind, genau eingegangen.
Dazu gehören der Aufbau der JSON Dateien für Level und Wunschkriterien, die im Pflichtenheft beschrieben, aber im Entwurf nicht umgesetzt wurden.
